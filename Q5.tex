%نام و نام خانوادگی:
%شماره دانشجویی: 
\مسئله{نام سؤال}

\پاسخ{}

گراف به صورت زیر است:
در اینجا d بیانگر رقم است.
\begin{latin}
    \begin{tikzpicture}
        \node[state,initial] (q0) {$q_0$};
        \node[state,right of=q0] (q1) {$q_1$};
        \node[state,right of=q1,accepting] (q2) {$q_2$};
        \node[state,below of=q0] (q3) {$q_3$};
        \node[state,right of=q3,accepting] (q4) {$q_4$};
        
        \draw (q0) edge[above] node{d} (q1);
        \draw (q1) edge[loop above] node{d} (q1);
        \draw (q1) edge[above] node{.} (q2);
        \draw (q2) edge[loop above] node{d} (q2);
        \draw (q0) edge[right] node{.} (q3);
        \draw (q3) edge[above] node{d} (q4);
        \draw (q4) edge[loop below] node{d} (q4);
    \end{tikzpicture}
\end{latin}
\newpage
کد به صورت زیر است:
\begin{latin}
    \begin{lstlisting}
#include <iostream>
#include <cctype>
using namespace std;

bool is_correct_float(const string& s){
    if (s == ".")
        return false;
    return all_of(s.begin(),s.end(),[] (const char c) -> bool {
        return isdigit(c) || c == '.';
    }) && any_of(s.begin(),s.end(),[] (const char c) -> bool {
        return c == '.';
    });
}
int main()
{
    string s = "2345678.";
    cout << (is_correct_float(s) ? "true" : "false") << endl;
    return 0;
}
    \end{lstlisting}
\end{latin}
