%نام و نام خانوادگی:
%شماره دانشجویی: 
\مسئله{نام سؤال}


\پاسخ{}

الف) خیر کامپایلر از زبان سطح پایینتر به بالاتر نیز )همان دیکامپایل( و همچنین کامپایلر های زبان نرم افزار به سخت افزار هم داریم.

ب) در زبان های پویا کامپایل کردن سریع تر و عملکرد در زمان اجرا کند تر است. \newline
این به خاطر آن است که کامپایلر برخی کار ها را به ماشین مجازی زمان اجرا می‌گذارد و خودش از انجام آن‌ها سر باز می‌زند. پس سرعت کامپایل بیشتر است. اما از طرفی دز زمان اجرا این زبان میانی است که باید به نحوی اجرا شود. که باعث می‌شود نیاز به مشاین مجازی داشته باشیم که این زبان میانی را به زبان قابل فهم برای سخت افزار تبدیل کند. این فرایند نیز خود زمان‌بر است و چیزی علاوه بر حالتی است که زبان ماشین حاضر و آماده است. پس زمان اجرا در این رویکرد بیشتر است و اجرا و تبدیل زبان میانی به ماشین کد کند تر است. \newline

ج) 

\begin{latin}
\begin{tikzpicture}

    \node[state,initial] (q0) {$q_0$};
    \node[state,right of=q0] (q1) {$q_1$};
    \node[state,below of=q1] (q2) {$q_2$};
    \node[state,right of=q1,accepting] (q3) {$q_3$};

    \draw (q0) edge[above] node{T\_"} (q1);
    \draw (q1) edge[bend right] node[left]{'\textbackslash'} (q2);
    \draw (q2) edge[bend right] node[right]{'n','t','"'} (q1);
    \draw (q1) edge[loop above] node{$\Sigma$ - \{'\textbackslash','"'\}} (q1);
    \draw (q1) edge[above] node{T\_"} (q3);

\end{tikzpicture}
\end{latin}
