%نام و نام خانوادگی:
%شماره دانشجویی: 
\مسئله{نام سؤال}

\پاسخ{}

الف) دیگر بین ضرب و جمع الویتی قائل نیستیم و هر کدام که زودتر بیاید آن را انجام می‌دهیم.

ب) 

\begin{latin}
    \begin{tikzpicture}[node distance=4.5cm]
        \node[state, initial] (q0) {$q_0$};
        \node[state, right of=q0,accepting] (q1) {$q_1$};
        \node[state, below of=q0] (q2) {$q_2$};
        \node[state, right of=q1] (q3) {$q_3$};
        \node[state, below of=q3] (q4) {$q_4$};
        
        \draw (q0) edge[above] node{T} (q1);
        \draw (q0) edge[left] node{-} (q2);
        \draw (q2) edge[left] node{T/@sub} (q1);
        \draw (q1) edge[above] node{+} (q3);
        \draw (q1) edge[above] node{-} (q4);
        \draw (q4) edge[bend left] node[left]{T/@sub} (q1);
        \draw (q3) edge[bend right] node[above]{T/@add} (q1);
        
    \end{tikzpicture}

    \begin{tikzpicture}
        \node[state,initial,initial text=T] (q0) {$q_0$};
        \node[state,accepting,right of=q0] (q1) {$q_1$};
        \node[state,right of=q1] (q2) {$q_2$};

        \draw (q0) edge[above] node{F} (q1);
        \draw (q1) edge[above] node{*} (q2);
        \draw (q2) edge[bend right=2cm] node[above]{F/@mult} (q1);
    \end{tikzpicture}

    \begin{tikzpicture}
        \node[state,initial,initial text=F] (q0) {$q_0$};
        \node[state,right of=q0,accepting] (q1) {$q_1$};
        \node[state,below of=q0] (q2) {$q_2$};
        \node[state,below of=q1] (q3) {$q_3$};
    
        \draw (q0) edge[above] node{id/@push} (q1);
        \draw (q0) edge[left] node{$($} (q2); 
        \draw (q2) edge[above] node{E} (q3);
        \draw (q3) edge[right] node{$)$} (q1);
    \end{tikzpicture}

\end{latin}
